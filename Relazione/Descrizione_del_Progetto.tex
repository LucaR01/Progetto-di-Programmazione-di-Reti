% ------------------ DESCRIZIONE DEL PROGETTO ----------------------------------

%\chapter{}

\section{Descrizione del Progetto}

\subsection{Introduzione}

\textsf{\normalsize Il progetto consiste nello sviluppo di un programma che permetta la connessione TCP client-server, per svolgere un mini-quiz tra più utenti. [Traccia 3]} \\

\subsection{Descrizione}

\textsf{\normalsize \textbf{Server:} è il tramite che instaura e permette la connessione tra gli utenti, giocatori. Si occupa di tutte le fasi di gioco.}\\

\textsf{\normalsize \textbf{Clients: } Sono gli utenti giocatori. Interagiscono col server per poter giocare.}\break

\textsf{\normalsize Il gioco comprende due fasi:}
\begin{itemize}
	\item \textsf{\normalsize Scelta tra 3 domande, di cui 1 a trabocchetto.}
	\item \textsf{\normalsize Risposta alla domanda scelta.}
\end{itemize}

\textsf{\normalsize Se l'utente avrà scelto la risposta giusta, otterrà un punto, altrimenti ne perderà uno.} \\
\textsf{\normalsize Invece se avrà scelto la domanda trabocchetto verrà eliminato.}\break

\textsf{\normalsize \textbf{Scopo del gioco:} Totalizzare più punti dei propri avversari.}\\

\subsection{Requisiti}

\subsubsection{Requisiti funzionali}

\begin{itemize}
	\item \textsf{\normalsize Possibilità di connettersi al server.}
	\item \textsf{\normalsize Possibilità di comunicare col server.}
	\item \textsf{\normalsize Possibilità di rispondere alle domande.}
	\item \textsf{\normalsize Avere un timer di gioco.}
	\item \textsf{\normalsize Avere delle domande trabocchetto che fanno eliminare automaticamente il giocatore.}
\end{itemize}

\subsubsection{Requisiti aggiuntivi | opzionali}
%\enlargethispage{1\linewidth}
\begin{itemize}
	\item \textsf{\normalsize Avere una classifica finale coi punteggi dei giocatori.}
	\begin{itemize}
		\item \textsf{\normalsize Incluso il numero di risposte corrette e sbagliate per ciascun giocatore.}
	\end{itemize}
	\item \textsf{\normalsize Possibilità di rigiocare una volta conclusa la partita.}
	\item \textsf{\normalsize Possibilità di poter uscire senza bloccare il server e comunque lasciando la possibilità agli altri utenti di poter rigiocare.}
\end{itemize}

%\newpage

\begin{comment} % questo lo metto in Comunicazione client-server
\subsection{Fasi di gioco}

\begin{itemize}
\item \textsf{\normalsize Gli utenti si connettono al server}
\item \textsf{\normalsize}
\item \textsf{\normalsize}
\item \textsf{\normalsize}
\end{itemize}
\end{comment}
