% --- Relazione Progetto di Programmazione di Reti ---
\documentclass[a4paper, 12pt]{article} % Qui era 12pt

% --- PACKAGES ---

\usepackage[italian]{babel}
\usepackage{comment}

\usepackage{microtype}
\usepackage{graphicx}
\usepackage{wrapfig}
\usepackage{enumitem}
\usepackage{fancyhdr}

\usepackage{amsmath}

%\usepackage{paralist} % SE ATTIVO PARALIST, ITEMIZE ED ENUMERATE NON FUNZIONANO!!!

\usepackage{amsthm}
% it introduces the \newtheorem command
\usepackage{amssymb}
% new mathematical symbols
\usepackage{eucal}
% other mathematical symbols
\usepackage{gensymb}
% \degree \celsius \micro \ohm \perthousand
\usepackage{mathptmx}
% other mathematical symbols
\usepackage{latexsym}
% other mathematical symbols
\usepackage{mathtools}
% supplements amsmath
\usepackage{textcomp}
% extra symbols arrows like \textrightarrow, \texteuro, \textcelsius

\usepackage[utf8]{inputenc}
%\usepackage[utf8x]{inputenc}
%\usepackage[latin1]{inputenc}
\usepackage{listingsutf8}
%\usepackage[overload]{textcase}

\usepackage{multicol}
% Per le colonne

%\usepackage{xcolor}
% Per colorare il testo

\usepackage[table, dvipsnames]{xcolor}
% Per ulteriori colori
% Quel table serve invece per colorare le righe in tabular

\usepackage[document]{ragged2e}

\usepackage{blindtext}

\usepackage[scaled=.92]{helvet}

\usepackage{tikz}

%\usepackage{floatrow}

%\usepackage{makecell}

\usepackage[T1]{fontenc}
%\usepackage{fourier, erewhon}
\usepackage{geometry}
\usepackage{array, caption, floatrow, tabularx, makecell, booktabs}%

\usepackage{amsfonts}

%\graphicspath{ {./images/} }
% graphicspath dice a LaTex la path della cartella dove si trovano le immagini.

%\usepackage{float}

\usepackage{etoc}

\usepackage{titlesec}
%\titleformat{\section}{\normalfont\Large\bfseries\center}{}{0pt}{} % Questo serve per togliere il numero dalla section ; Qui ho aggiunto \center e ora sono centrati

%\usetikzlibrary{snakes}

\usepackage[most]{tcolorbox}

\usepackage{wasysym} % per il Checkmark boxed

\usepackage{hyperref} % per i link e i checkbox

%\usepackage{pict2e}


%\begin{comment}

\titleformat
{\section} % command
[display] % shape
{\normalfont\Large\bfseries\center} % format
{} % label
{-2ex} % sep %0.5ex
{
	\rule{\textwidth}{1pt} % 1 pt
	\vspace{-2pt} % 1ex
	\centering
} % before-code
[
\vspace{-2ex}% -0.5ex valori originari
\rule{\textwidth}{0.3pt} % 0.3pt
] % after-code

%\end{comment}

\begin{comment}
\titleformat{\section}[block]
{\titlerule\addvspace{4pt}\normalfont\Large\bfseries\center}
{\thesection\enspace}{0pt}{}[\vspace{2pt}\titlerule]
\end{comment}

\titleformat{\subsection}{\normalfont\large\bfseries\center}{}{0pt}{} % Questo l'ho copiato da sopra ed ho aggiunto center e ora sono centrati
\titleformat{\subsubsection}{\normalfont\large\bfseries\center\color{red}}{}{0pt}{} % Questo l'ho copiato da sopra ed ho aggiunto center e ora sono centrati
\setcounter{secnumdepth}{2} % levels under \subsection are not numbered
% secnumdepth 1 è \section 2 è \subsection
\setcounter{tocdepth}{3}    % levels under \subsubsection are not listed in the TOC
% tocdepth 1 è section 2 è subsection e 3 è subsubsection

%\usepackage{sectsty} % Se lo attivo mi metti i numerini nella section
%\sectionfont{\centering}
%\allsectionsfont{\centering}

\usepackage{diffcoeff} % per le derivate

\usepackage{pagecolor,lipsum} %pagecolor per colorare la pagina (il background)
% lo sfondo in italiano
\usepackage{ltablex}

%\usepackage{multirow}

%\usepackage{systeme}

\usetikzlibrary{arrows.meta, angles}

\usetikzlibrary{chains,decorations.pathreplacing}

\usepackage[normalem]{ulem} % serve per sbarrare il testo (strikethrough), usare \sout

\usetikzlibrary{tikzmark} % per circondare gli elementi di una tabella (tabular)

\usetikzlibrary{shapes}

\usetikzlibrary{arrows,shapes.gates.logic.US,shapes.gates.logic.IEC,calc}

\tikzstyle{branch}=[fill,shape=circle,minimum size=3pt,inner sep=0pt]

\usepackage{subfig} % serve per avere immagini una accanto all'altra

\usepackage{cancel} 

\usetikzlibrary{matrix}

\usepackage{soul}

%\usepackage{emerald}% per dalla 1-4 firma (questo mi da errore, not found)

\usepackage{frcursive}% serve per la firma finale (nelle Conclusioni)

\usepackage{inslrmin}% per la 6 firma

\usepackage{pict2e}

%\usepackage{subcaption} 
% per le subfigures (più immagini una vicino all'altra)

%\usepackage{subfig}

\hypersetup{
	colorlinks=true,
	linkcolor=black, %blue
	filecolor=magenta,      
	urlcolor=cyan,
	pdftitle={Sharelatex Example},
	bookmarks=true,
	pdfpagemode=FullScreen,
}

%\usepackage{lscape}
%\usepackage{rotating}
%\usepackage{wrapfig} % per le figure attorno al testo


% --- FINE USEPACKAGES ---

% ============================ COMMANDS =========================================



% !TeX root = ../main.tex

% ============= FINE CUSTOM COMMANDS ============================================

\makeindex

%\begin{comment}
\renewcommand{\headrulewidth}{0pt}
\fancyhead[L]{}
\fancyhead[C]{ % o fancyhead
\includegraphics[width=5cm]{alma_mater_studiorum_cesena_logo.png}
}
\pagestyle{plain}
%\end{comment}

\begin{document}
	
%\begin{titlepage}
\newcommand{\HRule}{\rule{\linewidth}{0.5mm}} % Defines a new command for the horizontal lines, change thickness here

%\center % Center everything on the page

%----------------------------------------------------------------------------------------
%	HEADING SECTIONS
%----------------------------------------------------------------------------------------

\begin{comment}
\textsc{\LARGE University Name}\\[1.5cm] % Name of your university/college
\textsc{\Large Major Heading}\\[0.5cm] % Major heading such as course name
\textsc{\large Minor Heading}\\[0.5cm] % Minor heading such as course title
\end{comment}


%----------------------------------------------------------------------------------------
%	TITLE SECTION
%----------------------------------------------------------------------------------------

%\HRule \\[0.4cm]
%{ \huge \bfseries Title}\\[0.4cm] % Title of your document
%\HRule \\[1.5cm]
\begin{comment}
\textsf{%
\huge Relazione Progetto \\
\large di \\
\huge Programmazione di Reti \\}
\textrm{\large Luca Rengo \\
Matricola: 0000936177} \\
\textrm{\large Giugno | Luglio 2021}
\end{comment}


%----------------------------------------------------------------------------------------
%	AUTHOR SECTION
%----------------------------------------------------------------------------------------

\begin{comment}
\begin{minipage}{0.4\textwidth}
\begin{flushleft} \large
\emph{Author:}\\
John \textsc{Smith} % Your name
\end{flushleft}
\end{minipage}
~
\begin{minipage}{0.4\textwidth}
\begin{flushright} \large
\emph{Supervisor:} \\
Dr. James \textsc{Smith} % Supervisor's Name
\end{flushright}
\end{minipage}\\[2cm]
\end{comment}

% If you don't want a supervisor, uncomment the two lines below and remove the section above
%\Large \emph{Author:}\\
%John \textsc{Smith}\\[3cm] % Your name

%----------------------------------------------------------------------------------------
%	DATE SECTION
%----------------------------------------------------------------------------------------

%{\large \today}\\[2cm] % Date, change the \today to a set date if you want to be precise

%----------------------------------------------------------------------------------------
%	LOGO SECTION
%----------------------------------------------------------------------------------------

%\includegraphics[width = 9cm]{alma_mater_studiorum_cesena_logo.png}\\[1cm] % Include a department/university logo - this will require the graphicx package

%----------------------------------------------------------------------------------------

%\vfill % Fill the rest of the page with whitespace
%\end{titlepage}
	
	%\begin{comment}
	%\title{\huge{\textbf{Programmazione ad Oggetti}}}
	\title{%
		\huge Relazione Progetto \\
		\large di \\
		\huge Programmazione di Reti}
	\author{Luca Rengo \\
	Matricola: 0000936177}
	\date{Giugno | Luglio 2021}
	%\end{comment}
	%\includegraphics[width = 40mm]{alma_mater_studiorum_cesena_logo.png}
	
	% logo dell'università da aggiungere
	
	% il begin document forse dovrei metterlo qui.
	
	\maketitle	
	
	\thispagestyle{fancy}
	
	\pagebreak
	
	% INIZIO: TABLEOFCONTENTS
	
	\let\cleardoublepage\clearpage
	\tableofcontents
	\pagenumbering{arabic}
	\setcounter{page}{2}
	%\etocsettocstyle{\subsection*{This Chapter contains:}}{\noindent\rule{\linewidth}{.4pt}}
	% C'è bisogno del package etoc
	%\chapter{Extinct Species}
	%\localtableofcontents
	
	% FINE: TABLEOFCONTENTS
	
	\fancyhf{}
	\begin{comment}
	\renewcommand{\headrulewidth}{2pt}
	\renewcommand{\footrulewidth}{1pt}
	\fancyhead[LE]{\leftmark}
	\fancyhead[RO]{\nouppercase{\rightmark}}
	\fancyfoot[LE, RO]{\thepage}
	\end{comment}
	
	% descrizione del progetto, architettura dell'applicazione (design)/gui e Implementazione, configurazione e guida all'utilizzo
	% flow chart dell'applicazione.
	
	\include{Descrizione_del_progetto}
	
	% ------------- ARCHITETTURA DELL'APPLICAZIONE ----------------------------------

\section{Architettura e Design dell'Applicazione} % Architettura dell'Applicazione

\subsection{Comunicazione Client-Server}

\begin{itemize}
	\item \textsf{\normalsize Gli utenti si connettono al server e scelgono un nome.}
	%\item \textsf{\normalsize Gli utenti scelgono un nome.}
	\item \textsf{\normalsize Il server invia loro 3 domande (di cui 1 trabocchetto)}
	\item \textsf{\normalsize I clienti scelgono una tra le tre domande.}
	\begin{itemize}
		\item \textsf{\normalsize Se la domanda scelta è quella a trabocchetto, allora verranno eliminati e disconnessi dal server.}
		\item \textsf{\normalsize Altrimenti proseguono col gioco.}
	\end{itemize}
	\item \textsf{\normalsize Ricevono dal server 2 possibili risposte alla domanda da loro scelta.}
	\begin{itemize}
		\item \textsf{\normalsize Se rispondono correttamente , il loro punteggio (inizialmente a zero) salirà di 1.}
		\item \textsf{\normalsize Se sbagliano il loro punteggio verrà decrementato di 1.} % non verrà impattato (?)
	\end{itemize}
	\item \textsf{\normalsize Una volta risposto alla prima domanda (sia erratamente che correttamente) verrà fatto partire il timer.}
	\begin{itemize}
		\item \textsf{\normalsize Il timer durerà di default 30 secondi.}
	\end{itemize}
	\item \textsf{\normalsize Gli utenti continueranno, in questo modo, a ricevere domande e a dare risposte finchè il timer non terminerà.}%scadrà}
	\item \textsf{\normalsize Una volta che il timer avrà terminato, verrà segnalato al server di smettere di inviare domande.}
	\item \textsf{\normalsize Gli utenti , a questo punto, invieranno il loro punteggio al server.}
	\item \textsf{\normalsize Il server così decreterà il vincitore e i perdenti oppure il pareggio nel caso che tutti abbiano ottenuto lo stesso punteggio.}
	\item \textsf{\normalsize Dopodichè il Server invierà ai clients la classifica con i loro punteggi e quante risposte corrette e sbagliate hanno dato.}
	\item \textsf{\normalsize Conclusa la partita, il server resterà in attesa di notizie dai clients, per sapere se vogliono rigiocare oppure se si sono disconnessi.} % oppure se vogliono chiudere la connessione / uscire / smettere di giocare.
	%\item \textsf{\normalsize}
\end{itemize}

%\newpage

%\enlargethispage{1\linewidth}

\newgeometry{left = 1.2cm, right = 1.2cm}

\begin{figure}[p]
	\centering
	\includegraphics[width=1\linewidth]{./sequence_diagram_client-server_comunication_grid}
	\caption{Communicazione Client-Server Sequence Diagram}
	\label{fig:sequence_diagram_client-server_comunication}
\end{figure}

\restoregeometry

%\newpage

%\begin{comment}
\subsection{Il Client}
\textsf{\normalsize Il client si occupa di inizializzare un socket e di tutti gli aspetti che riguardano l'interfaccia grafica, in base alle direttive del server. Inoltre ogni client ha un proprio timer locale che parte in autonomia dal server, ma lo informa quando ha terminato.} \\

\subsection{UML Client}

%\newgeometry{left = .5cm , right = .5cm}

\begin{figure}[p] % p
	\centering
	\includegraphics[width=.8\linewidth]{./client_uml_transparent_background}
	\caption{Client UML}
	\label{fig:client_uml}
\end{figure}

%\restoregeometry

%\end{comment}
\textbf{\large Funzioni del cliente}
\begin{itemize}
	\item \textbf{\normalsize connect():}
	\begin{itemize}
		\item \textsf{\normalsize Questa funzione che viene chiamata dal cliente, premendo il pulsante connect, permette la connessione con il server.}
	\end{itemize}
	\item \textbf{\normalsize receive\_channel(client): }
	\begin{itemize}
		\item \textsf{\normalsize Questa è la funzione principale che prende come parametro un client (socket) e chiama la funzione recv\_initial\_questions(client)}
	\end{itemize}
	\item \textbf{\normalsize recv\_initial\_questions(client): }
	\begin{itemize}
		\item \textsf{\normalsize Questa funzione permette al cliente di ricevere le 3 domande (con 1 trabocchetto) dal server.}
	\end{itemize}
	\item \textbf{\normalsize receive\_questions(client): }
	\begin{itemize}
		\item \textsf{\normalsize Questa funzione permette al client di ricevere la possibili risposte alla domanda da lui/lei scelta.}
	\end{itemize}
	\item \textbf{\normalsize scelta(arg, client):}
	\begin{itemize}
		\item \textsf{\normalsize Questa funzione che viene chiamata dal client dopo aver premuto uno dei due pulsanti di risposta alla domanda; aggiorna il punteggio del giocatore in base all'arg che corrisponde ad una risposta alla domanda  e richiama recv\_inital\_questions(client) per poter farsi inviare dal server altre domande.}
	\end{itemize}
	\item \textbf{\normalsize timer(counter\_started, client): }
	\begin{itemize}
		\item \textsf{\normalsize Questa funzione viene chiamata la prima volta che viene chiamata la funzione scelta(arg, client). Questa fa partire un timer di gioco di 30 secondi (di default) se il contatore (counter\_started) non era già partito.}
	\end{itemize}
	\item \textbf{\normalsize end\_game(client): }
	\begin{itemize}
		\item \textsf{\normalsize Questa funzione invia al server le statistiche di gioco e riceve da esso la classifica con tutti i giocatori.}
	\end{itemize}
	\item \textbf{\normalsize leaderboard(client): }
	\begin{itemize}
		\item \textsf{\normalsize Questa funzione viene chiamata una volta che il timer si è fermato. Riceve dall'utente le statistiche riguardo alle risposte date corrette e sbagliate e crea la classifica di tutti i giocatori e la invia al client.}
	\end{itemize}
	\item \textbf{\normalsize reset(client): }
	\begin{itemize}
		\item \textsf{\normalsize Questa funzione resetta alcune variabili e liste per poter far ripartire il gioco e chiama receive\_channel(client) per ricevere nuove domande dal server.}
	\end{itemize}
	\item \textbf{\normalsize male(name): }
	\begin{itemize}
		\item \textsf{\normalsize Questa funzione prende un name, ovvero una stringa e restituisce \textbf{True} se il nome passato come argomento è maschile altrimenti \textbf{False} se è femminile.}
	\end{itemize}
	\item \textbf{\normalsize quit(client): }
	\begin{itemize}
		\item \textsf{\normalsize Questa funzione notifica il server che il client sta per uscire e poi termina il programma.}
	\end{itemize}
\end{itemize}

\textbf{\large Funzioni del cliente che riguardano la GUI (Interfaccia grafica)}
%\enlargethispage{1\linewidth}
\begin{itemize}
	\item \textbf{\normalsize disable\_initial\_buttons(todo)}
	\begin{itemize}
		\item \textsf{\normalsize Questa funzione abilita/disabilita i 3 pulsanti delle 3 domande in base al valore di todo (bool).}
	\end{itemize}
	\item \textbf{\normalsize disable\_buttons(todo)}
	\begin{itemize}
		\item \textsf{\normalsize Questa funziona, come quella sopra, abilita/disabilita i 2 pulsanti delle 2 possibili risposte alla domanda scelta dall'utente in base al valore di todo (bool).}
	\end{itemize}
	\item \textbf{\normalsize update\_gui()}
	\begin{itemize}
		\item \textsf{\normalsize Questa funzione crea alcuni frames e aggiorna il testo di alcuni labels (quello del nome e quello del ruolo).}
	\end{itemize}
	\item \textbf{\normalsize create\_label(frame, text)}
	\begin{itemize}
		\item \textsf{\normalsize Questa funzione crea un label e viene usata per la creazione della leaderboard (classifica) visto che il numero di giocatori non è conosciuto a priori.}
	\end{itemize}
	\item \textbf{\normalsize question\_choice(question\_picked, client)}
	\begin{itemize}
		\item \textsf{\normalsize Questa funzione valuta quale domanda abbia scelto l'utente e se era la domanda trabocchetto allora lo elimina dal gioco.}
	\end{itemize}
\end{itemize}

\newpage

\newgeometry{left=.8cm, right =.8cm}
%\enlargethispage{1\linewidth}
\subsection{Ordine delle Chiamate alle Funzioni del Cliente}
%\begin{landscape}
%\begin{sidewaysfigure}
%\textbf{\normalsize Ordine Chiamate Funzioni Cliente}
\begin{figure}[ht] % p
	\centering
	\includegraphics[width=1\linewidth]{./client_functions_call_order3}
	\caption{Diagramma di flusso: Ordine Chiamate Funzioni}
	\label{fig:client_functions_call_order}
\end{figure}
%\end{sidewaysfigure}
%\end{landscape}

\restoregeometry

\newpage

\subsection{Il Server}

\textsf{\normalsize Il server è il cuore portante dell'applicazione, fa partire il gioco, coordina i giocatori, stila una classifica ed informa i giocatori del risultato della partita.}

\subsection{Server UML}
%\newgeometry{left = 1cm, right = 1cm}
%\enlargethispage{1\linewidth}
\begin{figure}[p] % p
	\centering
	\includegraphics[width=.78\linewidth]{./server_uml}
	\caption{Server UML}
	\label{fig:server_uml}
\end{figure}

%\restoregeometry

\textbf{\normalsize Funzioni del Server}

\begin{itemize}
	\item \textbf{\normalsize accept\_connections()}
	\begin{itemize}
		\item \textsf{\normalsize Questa funzione viene chiamata da una thread ogni qual volta che un client si connette al server.}
	\end{itemize}
	\item \textbf{\normalsize pick\_role(nome)}
	\begin{itemize}
		\item \textsf{\normalsize Questa funzione restituisce un ruolo maschile o femminile in base al nome passato come parametro.}
	\end{itemize}
	\item \textbf{\normalsize pick\_questions(client\_connection)}
	\begin{itemize}
		\item \textsf{\normalsize Questa funzione sceglie delle domande casuali (tra cui una trabocchetto) li invia al client e poi aspetta di sapere dal client quale ha scelto per poter chiamare evaluate\_question(client\_connection) per poterla valutare.}
	\end{itemize}
	\item \textbf{\normalsize evaluate\_question(client\_connection)}
	\begin{itemize}
		\item \textsf{\normalsize Questa funzione valuta la domanda scelta dall'utente, se è la domanda trabocchetto informa l'utente che ha perso ed è stato eliminato, altrimenti prosegue con send\_question(client).}
	\end{itemize}
	\item \textbf{\normalsize send\_question(client)}
	\begin{itemize}
		\item \textsf{\normalsize Questa funzione invia all'utente la domanda che ha scelto più le 2 possibili risposte ad essa.}
	\end{itemize}
	\item \textbf{\normalsize gestisci\_utente(utente\_connection)}
	\begin{itemize}
		\item \textsf{\normalsize Questa è la funzione principale che viene chiamata una volta che l'utente ha acceduto al server. Si occupa di inviare/ricevere comunicazioni col client e creare la classifica e di terminare la partita.}
	\end{itemize}
	\item \textbf{\normalsize send\_leaderboard(utente\_connection)}
	\begin{itemize}
		\item \textsf{\normalsize Questa funzione si occupa di raggruppare le varie statistiche dai giocatori e creare la classifica ed inviarla ai clients.}
	\end{itemize}
	\item \textbf{\normalsize game\_over(utente\_connection)}
	\begin{itemize}
		\item \textsf{\normalsize Questa funzione chiama check\_client(client) per controllare se gli utenti sono ancora connessi e poi chiama reset(utente) per far ripartire il gioco.}
	\end{itemize}
	\item \textbf{\normalsize reset(utente)}
	\begin{itemize}
		\item \textsf{\normalsize Questa funzione si occupa di azzerare alcune variabili e liste per poter far ripartire il gioco da capo.}
	\end{itemize}
	\item \textbf{\normalsize close\_connection(utente\_connection)}
	\begin{itemize}
		\item \textsf{\normalsize Questa funzione si occupa di chiudere la connessione con un determinato client e di cancellarlo dalle liste in cui era presente.}
	\end{itemize}
	\item \textbf{\normalsize check\_client(client\_connection)}
	\begin{itemize}
		\item \textsf{\normalsize Questa funzione riceve un messaggio dal client e in base a ciò valuta se esso resterà connesso o no. Se intende interrompere la connessione chiama close\_connection(utente\_connection)}
	\end{itemize}
	\item \textbf{\normalsize evaluate\_winner(utente\_connection)}
	\begin{itemize}
		\item \textsf{\normalsize Questa funzione riceve dai clients i loro risultati e li riordina e decreta il vincitore/perdente/pareggio ed invia ad ogni client il proprio risultato.}
	\end{itemize}
\end{itemize}

\newpage
\newgeometry{left=.5cm, right =.5cm}
\subsection{Ordine delle Chiamate alle Funzioni del Server}
%\enlargethispage{1\baselineskip}
%\addtolength{\hoffset}{-4cm}

\begin{figure}[ht] % p
	\centering
	\includegraphics[width=1\linewidth]{./server_functions_call_order3}
	\caption{Diagramma di flusso: Ordine Chiamate Funzioni}
	\label{fig:server_functions_call_order}
\end{figure}

\restoregeometry

\newpage

%\addtolength{\hoffset}{4cm}


	
	% ----------- CONFIGURAZIONE E GUIDA ALL'UTILIZZO ------------------------------

\section{Configurazione e Guida all'Utilizzo}

\subsection{Configurazione}

\textsf{\normalsize E' possibile modificare alcune configurazioni del programma modificando alcune variabili: } \\

\textsf{\normalsize Nel file \textbf{game\_server}: } \\

\begin{itemize}
	\item \textbf{\small HOST}: L'indirizzo del server.
	\item \textbf{\small PORT}: La porta del server.
	\item \textbf{\small BUFSIZE}: La dimensione del buffer di ricezione dei messaggi dal cliente.
	\item \textbf{\small FORMAT}: E' possibile cambiare il FORMAT dei messaggi che è di default \textbf{utf-8}, ma andrebbe così cambiato anche nel client.
	\item \textbf{\small domande}: E' possibile aggiungere/rimuovere delle domande con le loro possibili risposte. (max 2 possibili risposte)
	\item \textbf{\small domande\_errate}: E' possibile aggiungere/rimuovere delle domande trabocchetto.
	\item \textbf{\small male\_roles}: E' possibile aggiungere/rimuovere ruoli maschili.
	\item \textbf{\small female\_roles}: E' possibile aggiungere/rimuovere ruoli femminili.
\end{itemize}

\textsf{\normalsize Nel file \textbf{game\_client}: } \\

\begin{itemize}
	\item \textbf{\small HOST}: L'indirizzo del server a cui ci si vuole connettere.
	\item \textbf{\small PORT}: La porta con cui ci si vuole connettere al server.
	\item \textbf{\small BUFSIZE}: La dimensione del buffer di ricezione dei messaggi dal server.
	\item \textbf{\small FORMAT}: E' possibile cambiare il FORMAT dei messaggi che è di default \textbf{utf-8}, ma andrebbe così cambiato anche nel server.
	\item \textbf{\small TOT\_GAME\_SECS}: Il numero di secondi che durerà la partita.
	%\item \textbf{\small risposte\_corrette}: Sono le identiche domande che ha il server, ma con la risposta giusta.
	\item \textbf{\small risposte\_corrette}: E' possibile modificare questo dizionario che contiene le stesse identiche domande del server, ma con solo la risposta giusta, casomai si volessero aggiungere/rimuovere domande.
\end{itemize}

%\newpage

\newgeometry{left = .5cm, right = .5cm}

\enlargethispage{1\linewidth}
% Guida all'utilizzo Possibilità 3
\subsection{Guida all'utilizzo}
\begin{minipage}{.45\textwidth}
	\begin{itemize}
		\item \textsf{\normalsize Aprire un terminale per il server ed eseguire il comando: \textbf{python game\_server}.}
		\item \textsf{\normalsize Dopodichè aprire almeno un altro terminale per almeno un client e digitare: \textbf{python game\_client}.}
		\item \textsf{\normalsize Con il client si aprirà un'interfaccia grafica che chiederà l'utente di inserire un nome di almeno un carattere.}
		\item \textsf{\normalsize Inserito il nome si prema il pulsante \bf{"Connect"}.}
		\item \textsf{\normalsize Dunque si vedranno, sotto il nome, 3 pulsanti con 3 domande, l'utente ne scelga uno.}
		\item \textsf{\normalsize Se ha scelto la domanda trabocchetto, la GUI mostrerà \emph{"Sei stato/a eliminato/a!"}, dopodichè si aggiornerà di nuovo e mostrerà \emph{"Exiting.."} e dopo tot secondi uscirà dall'applicazione.}
		\item \textsf{\normalsize Se, invece, si è scelta una domanda corretta allora, sotto quei 3 pulsanti delle domande, si vedrà un label con la domanda scelta e 2 altri pulsanti di cui l'utente dovrà sceglierne uno.}
		\item \textsf{\normalsize Se l'utente ha scelto la risposta esatta, il suo contatore \emph{"Punteggio"} incrementerà di 1 altrimenti resterà al valore che aveva.}
	\end{itemize}
\end{minipage}
\begin{minipage}{.534\linewidth} % se metto newgeometry allora qui devo mettere .55\linewidth altrimenti sarebbe .75\linewidth
	\begin{minipage}{.45\linewidth}
		\begin{figure}[H]
			\includegraphics[width=\linewidth]{player_gui_connect.png}
			%\caption{Ciao}
		\end{figure}
	\end{minipage}
	\hspace{0.05\linewidth}
	\begin{comment}
	\begin{minipage}{.45\linewidth}
	\begin{figure}[H]
	\includegraphics[width=.75\linewidth]{player_gui_tre_domande2.png}
	%\caption{Ciao}
	\end{figure}
	\end{minipage}
	\end{comment}
	\hspace{0.05\linewidth}
	\begin{minipage}{.45\linewidth}
		\begin{figure}[H]
			\includegraphics[width=\linewidth]{player_gui_domanda_errata2}
		\end{figure}
	\end{minipage}
	\hspace{0.05\linewidth}
	\begin{minipage}{.45\linewidth}
		\begin{figure}[H]
			\includegraphics[width=\linewidth]{player_gui_exiting2}
		\end{figure}
	\end{minipage}
	\hspace{0.05\linewidth}
	\begin{minipage}{.45\linewidth}
		\begin{figure}[H]
			\includegraphics[width=\linewidth]{player_gui_match1_player1}
		\end{figure}
	\end{minipage}
	\hspace{0.05\linewidth}
	\begin{minipage}{.45\linewidth}
		\begin{figure}[H]
			\includegraphics[width=\linewidth]{player_gui_match1_player2}
		\end{figure}
	\end{minipage}
	%\hspace{0.05\linewidth}
\end{minipage}

\restoregeometry

\newpage

\newgeometry{left = .5cm, right = .5cm}

\begin{minipage}{.45\linewidth}
	\begin{itemize}
		\item \textsf{\normalsize Dopo aver risposto alla prima domanda, partirà un timer e il giocatore continuerà a ricevere domande e a rispondere finchè quel timer non si fermerà.}
		\item \textsf{\normalsize Una volta che il timer avrà concluso, l'utente resterà in attesa dal server del punteggio finale e della classifica.}
		\item \textsf{\normalsize Terminata la partita l'utente potrà cliccare su due pulsanti: \textbf{"RESTART"} per poter rigiocare o \textbf{"QUIT"} per poter uscire.}
		\item \textsf{\normalsize C'è anche la possibilità di pareggio.}
	\end{itemize}
\end{minipage}
\begin{minipage}{.55\linewidth} % se metto newgeometry allora qui metto .55, altrimenti sarebbe .75\linewidth
	\begin{minipage}{.45\linewidth}
		\begin{figure}[H]
			\includegraphics[width=\linewidth]{player_gui_match2_player1}
		\end{figure}
	\end{minipage}
	\hspace{0.05\linewidth}
	\begin{minipage}{.45\linewidth}
		\begin{figure}[H]
			\includegraphics[width=\linewidth]{player_gui_match2_player2}
		\end{figure}
	\end{minipage}
\end{minipage}

\restoregeometry

%\enlargethispage{1\linewidth}
% Guida all'utilizzo Possibilità 1
\begin{comment}
\begin{itemize}
	\item \textsf{\normalsize Aprire un terminale per il server ed eseguire il comando: %e digitare \textbf{python game\_server}.}
	\item \textsf{\normalsize Dopodichè aprire almeno un'altro terminale per almeno un client e digitare: \textbf{python game\_client}.}
	\item \textsf{\normalsize Con il client si aprirà un'interfaccia grafica che chiederà l'utente di inserire un nome di almeno un carattere.}
	\item \textsf{\normalsize Inserito il nome si prema il pulsante \bf{"Connect"}}
	\item \textsf{\normalsize Dunque si vedranno, sotto il nome, 3 pulsanti con 3 domande, l'utente ne scelga uno.}
	\item \textsf{\normalsize Se ha scelto la domanda trabocchetto, la GUI mostrerà \emph{"Sei stato/a eliminato/a!"}, dopodichè si aggiornerà di nuovo e mostrerà \emph{"Exiting.."} e dopo tot secondi uscirà dall'applicazione.}
	\item \textsf{\normalsize Se, invece, si è scelta una domanda corretta allora, sotto quei 3 pulsanti delle domande, si vedrà un label con la domanda scelta e 2 altri pulsanti di cui l'utente dovrà sceglierne uno.}
	\item \textsf{\normalsize Se l'utente ha scelto la risposta esatta, il suo contatore \emph{"Punteggio"} incrementerà di 1 altrimenti resterà al valore che aveva.}
	\item \textsf{\normalsize Dopo aver risposto alla prima domanda, partirà un timer e il giocatore continuerà a ricevere domande e a rispondere finchè quel timer non si fermerà.}
	\item \textsf{\normalsize Una volta che il timer avrà concluso, l'utente resterà in attesa dal server del punteggio finale e della classifica.}
	\item \textsf{\normalsize Terminata la partita l'utente potrà cliccare su due pulsanti: \textbf{"RESTART"} per poter rigiocare o \textbf{"QUIT"} per poter uscire.}
\end{itemize}
\end{comment}

\newpage

\begin{comment}
\begin{itemize}
\item 
\parbox[t]{\dimexpr\textwidth-\leftmargin}{%
\vspace{-2.5mm}
\begin{wrapfigure}[10]{r}{0.5\textwidth}
\centering
\vspace{-\baselineskip}
\includegraphics[width=.5\linewidth]{player_gui_connect.png}
\caption{The beach}
\end{wrapfigure}
\lipsum[1]
}
\item
\parbox[t]{\dimexpr\textwidth-\leftmargin}{%
\vspace{-2.5mm}
\begin{wrapfigure}{r}{0.5\textwidth}
\centering
\vspace{-\baselineskip}
\includegraphics[width=.5\linewidth]{player_gui_tre_domande.png}
\caption{The beach}
\end{wrapfigure}
This is a short text...
}
\item     \parbox[t]{\dimexpr\textwidth-\leftmargin}{%
\vspace{-2.5mm}
\begin{wrapfigure}[8]{r}{0.5\textwidth}
\end{wrapfigure}
\lipsum[1]}
\end{itemize}
\end{comment}

\begin{comment}
{\bfseries Multiple Choice: }% do not use \bf in LaTeX it is deprecated 20+ years ago
%\begin{enumerate}
%\item As shown in the figure. This rock is phaneritic and contains quartz, K-feldspar, and plagioclase in nearly equal amounts. What is it?

\begin{minipage}{.45\textwidth}
\begin{itemize}% never number things manually!
\item rhyolite
\item basa
\item diorite
\item ash-flow tuff
\item  granite
\end{itemize}
\end{minipage}
\begin{minipage}{.45\textwidth}
\includegraphics[width=1\linewidth]{player_gui_connect.png}
\end{minipage}
%\end{enumerate}
\end{comment}

% Guida all'utilizzo Possibilità 2
\begin{comment}
\begin{minipage}{.45\textwidth}
\begin{itemize}
	\item \textsf{\normalsize Aprire un terminale per il server ed eseguire il comando: \textbf{python game\_server}.} 
	%digitare \textbf{python game\_server}}
	\item \textsf{\normalsize Dopodichè aprire almeno un'altro terminale per almeno un client e digitare: \textbf{python game\_client}}
	\item \textsf{\normalsize Con il client si aprirà un'interfaccia grafica che chiederà l'utente di inserire un nome di almeno un carattere.}
	\item \textsf{\normalsize Inserito il nome si prema il pulsante \bf{"Connect"}}
\end{itemize}
\end{minipage}
\begin{minipage}{.45\textwidth}
	\includegraphics[width=.7\linewidth]{player_gui_connect.png}
\end{minipage}

\begin{minipage}{.45\linewidth}
\begin{itemize}
\item \textsf{\normalsize Dunque si vedranno, sotto il nome, 3 pulsanti con 3 domande, l'utente ne scelga uno.}
\item \textsf{\normalsize Se ha scelto la domanda trabocchetto, la GUI mostrerà \emph{"Sei stato/a eliminato/a!"}, dopodichè si aggiornerà di nuovo e mostrerà \emph{"Exiting.."} e dopo tot secondi uscirà dall'applicazione.}
\end{itemize}
\end{minipage}
\begin{minipage}{.45\linewidth}
	\includegraphics[width=.75\linewidth]{player_gui_tre_domande2.png}
\end{minipage}

\begin{minipage}{.45\linewidth}
\begin{itemize}
	\item \textsf{\normalsize Se, invece, si è scelta una domanda corretta allora, sotto quei 3 pulsanti delle domande, si vedrà un label con la domanda scelta e 2 altri pulsanti di cui l'utente dovrà sceglierne uno.}
	\item \textsf{\normalsize Se l'utente ha scelto la risposta esatta, il suo contatore \emph{Punteggio} incrementerà di 1 altrimenti resterà al valore che aveva.}
	\item \textsf{\normalsize Dopo aver risposto alla prima domanda, partirà un timer e il giocatore continuerà a ricevere domande e a rispondere finchè quel timer non si fermerà.}
\end{itemize}
\end{minipage}
\begin{minipage}{.7\linewidth}
	\begin{minipage}{0.45\linewidth}
		\begin{figure}[H]
			\includegraphics[width=\linewidth]{player_gui_domanda_errata2}
			\caption{This is the first figure}
		\end{figure}
	\end{minipage}
	\hspace{0.05\linewidth}
	\begin{minipage}{0.45\linewidth}
		\begin{figure}[H]
			\includegraphics[width=\linewidth]{player_gui_exiting2}
			\caption{This is the second figure}
		\end{figure}
	\end{minipage}
\end{minipage}

\begin{minipage}{.45\linewidth}
\begin{itemize}
	\item \textsf{\normalsize Una volta che il timer avrà concluso, l'utente resterà in attesa dal server del punteggio finale e della classifica.}
	\item \textsf{\normalsize Terminata la partita l'utente potrà cliccare su due pulsanti: \textbf{"RESTART"} per poter rigiocare o \textbf{"QUIT"} per poter uscire.}
\end{itemize}	
\end{minipage}
\begin{minipage}{.7\linewidth}
	\begin{minipage}{0.45\linewidth}
		\begin{figure}[H]
			\includegraphics[width=\linewidth]{player_gui_match1_player1}
			\caption{This is the first figure}
		\end{figure}
	\end{minipage}
	\hspace{0.05\linewidth}
	\begin{minipage}{0.45\linewidth}
		\begin{figure}[H]
			\includegraphics[width=\linewidth]{player_gui_match1_player2}
			\caption{This is the second figure}
		\end{figure}
	\end{minipage}
\end{minipage}

\begin{minipage}{.45\linewidth}
\begin{itemize}
	\item \textsf{\normalsize C'è anche la possibilità di pareggio.}
\end{itemize}
\end{minipage}
\begin{minipage}{.7\linewidth}
	\begin{minipage}{0.45\linewidth}
		\begin{figure}[H]
			\includegraphics[width=\linewidth]{player_gui_match2_player1}
			\caption{This is the first figure}
		\end{figure}
	\end{minipage}
	\hspace{0.05\linewidth}
	\begin{minipage}{0.45\linewidth}
		\begin{figure}[H]
			\includegraphics[width=\linewidth]{player_gui_match2_player2}
			\caption{This is the second figure}
		\end{figure}
	\end{minipage}
\end{minipage}
\end{comment}

\begin{comment}
\begin{figure}
\begin{subfigure}[b]{.3\textwidth}
\includegraphics[width=\textwidth]{player_gui_domanda_errata2.png}
\end{subfigure}
\begin{subfigure}[b]{.3\textwidth}
\includegraphics[width=\textwidth]{player_gui_exiting2.png}
\end{subfigure}
\end{figure}
\end{comment}

\begin{comment}
\begin{figure}
\begin{subfigure}[t]{0.475\textwidth}
\includegraphics[width=.3\textwidth]{player_gui_domanda_errata2}
\caption{Modello compartimentale mammellare (o mammillare).}
\label{fig-a}
\end{subfigure}\hfill
\begin{subfigure}[t]{0.475\textwidth}
\includegraphics[width=.3\textwidth]{player_gui_exiting2}
\caption{Modello compartimentale catenario.}
\label{fig-b}
\end{subfigure}
\caption{Principali topologie dei modelli compartimentali.} 
\label{fig:main}
\end{figure}
\end{comment}

% Ho usato questo!
\begin{comment}
\begin{minipage}{\linewidth}
\centering
\begin{minipage}{0.45\linewidth}
\begin{figure}[H]
\includegraphics[width=\linewidth]{player_gui_domanda_errata2}
\caption{This is the first figure}
\end{figure}
\end{minipage}
\hspace{0.05\linewidth}
\begin{minipage}{0.45\linewidth}
\begin{figure}[H]
\includegraphics[width=\linewidth]{player_gui_exiting2}
\caption{This is the second figure}
\end{figure}
\end{minipage}
\end{minipage}
\end{comment}
%\enlargethispage{1\linewidth}
	
	%\includeonly{}
	
\end{document}