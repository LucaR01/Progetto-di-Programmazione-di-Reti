% ----------- CONFIGURAZIONE E GUIDA ALL'UTILIZZO ------------------------------

\section{Configurazione e Guida all'Utilizzo}

\subsection{Configurazione}

\textsf{\normalsize E' possibile modificare alcune configurazioni del programma modificando alcune variabili: } \\

\textsf{\normalsize Nel file \textbf{game\_server}: } \\

\begin{itemize}
	\item \textbf{\small HOST}: L'indirizzo del server.
	\item \textbf{\small PORT}: La porta del server.
	\item \textbf{\small BUFSIZE}: La dimensione del buffer di ricezione dei messaggi dal cliente.
	\item \textbf{\small FORMAT}: E' possibile cambiare il FORMAT dei messaggi che è di default \textbf{utf-8}, ma andrebbe così cambiato anche nel client.
	\item \textbf{\small domande}: E' possibile aggiungere/rimuovere delle domande con le loro possibili risposte. (max 2 possibili risposte)
	\item \textbf{\small domande\_errate}: E' possibile aggiungere/rimuovere delle domande trabocchetto.
	\item \textbf{\small male\_roles}: E' possibile aggiungere/rimuovere ruoli maschili.
	\item \textbf{\small female\_roles}: E' possibile aggiungere/rimuovere ruoli femminili.
\end{itemize}

\textsf{\normalsize Nel file \textbf{game\_client}: } \\

\begin{itemize}
	\item \textbf{\small HOST}: L'indirizzo del server a cui ci si vuole connettere.
	\item \textbf{\small PORT}: La porta con cui ci si vuole connettere al server.
	\item \textbf{\small BUFSIZE}: La dimensione del buffer di ricezione dei messaggi dal server.
	\item \textbf{\small FORMAT}: E' possibile cambiare il FORMAT dei messaggi che è di default \textbf{utf-8}, ma andrebbe così cambiato anche nel server.
	\item \textbf{\small TOT\_GAME\_SECS}: Il numero di secondi che durerà la partita.
	%\item \textbf{\small risposte\_corrette}: Sono le identiche domande che ha il server, ma con la risposta giusta.
	\item \textbf{\small risposte\_corrette}: E' possibile modificare questo dizionario che contiene le stesse identiche domande del server, ma con solo la risposta giusta, casomai si volessero aggiungere/rimuovere domande.
\end{itemize}

%\newpage

\newgeometry{left = .5cm, right = .5cm}

\enlargethispage{1\linewidth}
% Guida all'utilizzo Possibilità 3
\subsection{Guida all'utilizzo}
\begin{minipage}{.45\textwidth}
	\begin{itemize}
		\item \textsf{\normalsize Aprire un terminale per il server ed eseguire il comando: \textbf{python game\_server}.}
		\item \textsf{\normalsize Dopodichè aprire almeno un altro terminale per almeno un client e digitare: \textbf{python game\_client}.}
		\item \textsf{\normalsize Con il client si aprirà un'interfaccia grafica che chiederà l'utente di inserire un nome di almeno un carattere.}
		\item \textsf{\normalsize Inserito il nome si prema il pulsante \bf{"Connect"}.}
		\item \textsf{\normalsize Dunque si vedranno, sotto il nome, 3 pulsanti con 3 domande, l'utente ne scelga uno.}
		\item \textsf{\normalsize Se ha scelto la domanda trabocchetto, la GUI mostrerà \emph{"Sei stato/a eliminato/a!"}, dopodichè si aggiornerà di nuovo e mostrerà \emph{"Exiting.."} e dopo tot secondi uscirà dall'applicazione.}
		\item \textsf{\normalsize Se, invece, si è scelta una domanda corretta allora, sotto quei 3 pulsanti delle domande, si vedrà un label con la domanda scelta e 2 altri pulsanti di cui l'utente dovrà sceglierne uno.}
		\item \textsf{\normalsize Se l'utente ha scelto la risposta esatta, il suo contatore \emph{"Punteggio"} incrementerà di 1 altrimenti resterà al valore che aveva.}
	\end{itemize}
\end{minipage}
\begin{minipage}{.534\linewidth} % se metto newgeometry allora qui devo mettere .55\linewidth altrimenti sarebbe .75\linewidth
	\begin{minipage}{.45\linewidth}
		\begin{figure}[H]
			\includegraphics[width=\linewidth]{player_gui_connect.png}
			%\caption{Ciao}
		\end{figure}
	\end{minipage}
	\hspace{0.05\linewidth}
	\begin{comment}
	\begin{minipage}{.45\linewidth}
	\begin{figure}[H]
	\includegraphics[width=.75\linewidth]{player_gui_tre_domande2.png}
	%\caption{Ciao}
	\end{figure}
	\end{minipage}
	\end{comment}
	\hspace{0.05\linewidth}
	\begin{minipage}{.45\linewidth}
		\begin{figure}[H]
			\includegraphics[width=\linewidth]{player_gui_domanda_errata2}
		\end{figure}
	\end{minipage}
	\hspace{0.05\linewidth}
	\begin{minipage}{.45\linewidth}
		\begin{figure}[H]
			\includegraphics[width=\linewidth]{player_gui_exiting2}
		\end{figure}
	\end{minipage}
	\hspace{0.05\linewidth}
	\begin{minipage}{.45\linewidth}
		\begin{figure}[H]
			\includegraphics[width=\linewidth]{player_gui_match1_player1}
		\end{figure}
	\end{minipage}
	\hspace{0.05\linewidth}
	\begin{minipage}{.45\linewidth}
		\begin{figure}[H]
			\includegraphics[width=\linewidth]{player_gui_match1_player2}
		\end{figure}
	\end{minipage}
	%\hspace{0.05\linewidth}
\end{minipage}

\restoregeometry

\newpage

\newgeometry{left = .5cm, right = .5cm}

\begin{minipage}{.45\linewidth}
	\begin{itemize}
		\item \textsf{\normalsize Dopo aver risposto alla prima domanda, partirà un timer e il giocatore continuerà a ricevere domande e a rispondere finchè quel timer non si fermerà.}
		\item \textsf{\normalsize Una volta che il timer avrà concluso, l'utente resterà in attesa dal server del punteggio finale e della classifica.}
		\item \textsf{\normalsize Terminata la partita l'utente potrà cliccare su due pulsanti: \textbf{"RESTART"} per poter rigiocare o \textbf{"QUIT"} per poter uscire.}
		\item \textsf{\normalsize C'è anche la possibilità di pareggio.}
	\end{itemize}
\end{minipage}
\begin{minipage}{.55\linewidth} % se metto newgeometry allora qui metto .55, altrimenti sarebbe .75\linewidth
	\begin{minipage}{.45\linewidth}
		\begin{figure}[H]
			\includegraphics[width=\linewidth]{player_gui_match2_player1}
		\end{figure}
	\end{minipage}
	\hspace{0.05\linewidth}
	\begin{minipage}{.45\linewidth}
		\begin{figure}[H]
			\includegraphics[width=\linewidth]{player_gui_match2_player2}
		\end{figure}
	\end{minipage}
\end{minipage}

\restoregeometry

%\enlargethispage{1\linewidth}
% Guida all'utilizzo Possibilità 1
\begin{comment}
\begin{itemize}
	\item \textsf{\normalsize Aprire un terminale per il server ed eseguire il comando: %e digitare \textbf{python game\_server}.}
	\item \textsf{\normalsize Dopodichè aprire almeno un'altro terminale per almeno un client e digitare: \textbf{python game\_client}.}
	\item \textsf{\normalsize Con il client si aprirà un'interfaccia grafica che chiederà l'utente di inserire un nome di almeno un carattere.}
	\item \textsf{\normalsize Inserito il nome si prema il pulsante \bf{"Connect"}}
	\item \textsf{\normalsize Dunque si vedranno, sotto il nome, 3 pulsanti con 3 domande, l'utente ne scelga uno.}
	\item \textsf{\normalsize Se ha scelto la domanda trabocchetto, la GUI mostrerà \emph{"Sei stato/a eliminato/a!"}, dopodichè si aggiornerà di nuovo e mostrerà \emph{"Exiting.."} e dopo tot secondi uscirà dall'applicazione.}
	\item \textsf{\normalsize Se, invece, si è scelta una domanda corretta allora, sotto quei 3 pulsanti delle domande, si vedrà un label con la domanda scelta e 2 altri pulsanti di cui l'utente dovrà sceglierne uno.}
	\item \textsf{\normalsize Se l'utente ha scelto la risposta esatta, il suo contatore \emph{"Punteggio"} incrementerà di 1 altrimenti resterà al valore che aveva.}
	\item \textsf{\normalsize Dopo aver risposto alla prima domanda, partirà un timer e il giocatore continuerà a ricevere domande e a rispondere finchè quel timer non si fermerà.}
	\item \textsf{\normalsize Una volta che il timer avrà concluso, l'utente resterà in attesa dal server del punteggio finale e della classifica.}
	\item \textsf{\normalsize Terminata la partita l'utente potrà cliccare su due pulsanti: \textbf{"RESTART"} per poter rigiocare o \textbf{"QUIT"} per poter uscire.}
\end{itemize}
\end{comment}

\newpage

\begin{comment}
\begin{itemize}
\item 
\parbox[t]{\dimexpr\textwidth-\leftmargin}{%
\vspace{-2.5mm}
\begin{wrapfigure}[10]{r}{0.5\textwidth}
\centering
\vspace{-\baselineskip}
\includegraphics[width=.5\linewidth]{player_gui_connect.png}
\caption{The beach}
\end{wrapfigure}
\lipsum[1]
}
\item
\parbox[t]{\dimexpr\textwidth-\leftmargin}{%
\vspace{-2.5mm}
\begin{wrapfigure}{r}{0.5\textwidth}
\centering
\vspace{-\baselineskip}
\includegraphics[width=.5\linewidth]{player_gui_tre_domande.png}
\caption{The beach}
\end{wrapfigure}
This is a short text...
}
\item     \parbox[t]{\dimexpr\textwidth-\leftmargin}{%
\vspace{-2.5mm}
\begin{wrapfigure}[8]{r}{0.5\textwidth}
\end{wrapfigure}
\lipsum[1]}
\end{itemize}
\end{comment}

\begin{comment}
{\bfseries Multiple Choice: }% do not use \bf in LaTeX it is deprecated 20+ years ago
%\begin{enumerate}
%\item As shown in the figure. This rock is phaneritic and contains quartz, K-feldspar, and plagioclase in nearly equal amounts. What is it?

\begin{minipage}{.45\textwidth}
\begin{itemize}% never number things manually!
\item rhyolite
\item basa
\item diorite
\item ash-flow tuff
\item  granite
\end{itemize}
\end{minipage}
\begin{minipage}{.45\textwidth}
\includegraphics[width=1\linewidth]{player_gui_connect.png}
\end{minipage}
%\end{enumerate}
\end{comment}

% Guida all'utilizzo Possibilità 2
\begin{comment}
\begin{minipage}{.45\textwidth}
\begin{itemize}
	\item \textsf{\normalsize Aprire un terminale per il server ed eseguire il comando: \textbf{python game\_server}.} 
	%digitare \textbf{python game\_server}}
	\item \textsf{\normalsize Dopodichè aprire almeno un'altro terminale per almeno un client e digitare: \textbf{python game\_client}}
	\item \textsf{\normalsize Con il client si aprirà un'interfaccia grafica che chiederà l'utente di inserire un nome di almeno un carattere.}
	\item \textsf{\normalsize Inserito il nome si prema il pulsante \bf{"Connect"}}
\end{itemize}
\end{minipage}
\begin{minipage}{.45\textwidth}
	\includegraphics[width=.7\linewidth]{player_gui_connect.png}
\end{minipage}

\begin{minipage}{.45\linewidth}
\begin{itemize}
\item \textsf{\normalsize Dunque si vedranno, sotto il nome, 3 pulsanti con 3 domande, l'utente ne scelga uno.}
\item \textsf{\normalsize Se ha scelto la domanda trabocchetto, la GUI mostrerà \emph{"Sei stato/a eliminato/a!"}, dopodichè si aggiornerà di nuovo e mostrerà \emph{"Exiting.."} e dopo tot secondi uscirà dall'applicazione.}
\end{itemize}
\end{minipage}
\begin{minipage}{.45\linewidth}
	\includegraphics[width=.75\linewidth]{player_gui_tre_domande2.png}
\end{minipage}

\begin{minipage}{.45\linewidth}
\begin{itemize}
	\item \textsf{\normalsize Se, invece, si è scelta una domanda corretta allora, sotto quei 3 pulsanti delle domande, si vedrà un label con la domanda scelta e 2 altri pulsanti di cui l'utente dovrà sceglierne uno.}
	\item \textsf{\normalsize Se l'utente ha scelto la risposta esatta, il suo contatore \emph{Punteggio} incrementerà di 1 altrimenti resterà al valore che aveva.}
	\item \textsf{\normalsize Dopo aver risposto alla prima domanda, partirà un timer e il giocatore continuerà a ricevere domande e a rispondere finchè quel timer non si fermerà.}
\end{itemize}
\end{minipage}
\begin{minipage}{.7\linewidth}
	\begin{minipage}{0.45\linewidth}
		\begin{figure}[H]
			\includegraphics[width=\linewidth]{player_gui_domanda_errata2}
			\caption{This is the first figure}
		\end{figure}
	\end{minipage}
	\hspace{0.05\linewidth}
	\begin{minipage}{0.45\linewidth}
		\begin{figure}[H]
			\includegraphics[width=\linewidth]{player_gui_exiting2}
			\caption{This is the second figure}
		\end{figure}
	\end{minipage}
\end{minipage}

\begin{minipage}{.45\linewidth}
\begin{itemize}
	\item \textsf{\normalsize Una volta che il timer avrà concluso, l'utente resterà in attesa dal server del punteggio finale e della classifica.}
	\item \textsf{\normalsize Terminata la partita l'utente potrà cliccare su due pulsanti: \textbf{"RESTART"} per poter rigiocare o \textbf{"QUIT"} per poter uscire.}
\end{itemize}	
\end{minipage}
\begin{minipage}{.7\linewidth}
	\begin{minipage}{0.45\linewidth}
		\begin{figure}[H]
			\includegraphics[width=\linewidth]{player_gui_match1_player1}
			\caption{This is the first figure}
		\end{figure}
	\end{minipage}
	\hspace{0.05\linewidth}
	\begin{minipage}{0.45\linewidth}
		\begin{figure}[H]
			\includegraphics[width=\linewidth]{player_gui_match1_player2}
			\caption{This is the second figure}
		\end{figure}
	\end{minipage}
\end{minipage}

\begin{minipage}{.45\linewidth}
\begin{itemize}
	\item \textsf{\normalsize C'è anche la possibilità di pareggio.}
\end{itemize}
\end{minipage}
\begin{minipage}{.7\linewidth}
	\begin{minipage}{0.45\linewidth}
		\begin{figure}[H]
			\includegraphics[width=\linewidth]{player_gui_match2_player1}
			\caption{This is the first figure}
		\end{figure}
	\end{minipage}
	\hspace{0.05\linewidth}
	\begin{minipage}{0.45\linewidth}
		\begin{figure}[H]
			\includegraphics[width=\linewidth]{player_gui_match2_player2}
			\caption{This is the second figure}
		\end{figure}
	\end{minipage}
\end{minipage}
\end{comment}

\begin{comment}
\begin{figure}
\begin{subfigure}[b]{.3\textwidth}
\includegraphics[width=\textwidth]{player_gui_domanda_errata2.png}
\end{subfigure}
\begin{subfigure}[b]{.3\textwidth}
\includegraphics[width=\textwidth]{player_gui_exiting2.png}
\end{subfigure}
\end{figure}
\end{comment}

\begin{comment}
\begin{figure}
\begin{subfigure}[t]{0.475\textwidth}
\includegraphics[width=.3\textwidth]{player_gui_domanda_errata2}
\caption{Modello compartimentale mammellare (o mammillare).}
\label{fig-a}
\end{subfigure}\hfill
\begin{subfigure}[t]{0.475\textwidth}
\includegraphics[width=.3\textwidth]{player_gui_exiting2}
\caption{Modello compartimentale catenario.}
\label{fig-b}
\end{subfigure}
\caption{Principali topologie dei modelli compartimentali.} 
\label{fig:main}
\end{figure}
\end{comment}

% Ho usato questo!
\begin{comment}
\begin{minipage}{\linewidth}
\centering
\begin{minipage}{0.45\linewidth}
\begin{figure}[H]
\includegraphics[width=\linewidth]{player_gui_domanda_errata2}
\caption{This is the first figure}
\end{figure}
\end{minipage}
\hspace{0.05\linewidth}
\begin{minipage}{0.45\linewidth}
\begin{figure}[H]
\includegraphics[width=\linewidth]{player_gui_exiting2}
\caption{This is the second figure}
\end{figure}
\end{minipage}
\end{minipage}
\end{comment}
%\enlargethispage{1\linewidth}